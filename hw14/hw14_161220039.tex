%!TEX program = xelatex
\documentclass[a4papers]{ctexart}
%数学符号
\usepackage{amssymb}
\usepackage{amsmath}
%表格
\usepackage{graphicx,floatrow}
\usepackage{array}
\usepackage{booktabs}
\usepackage{makecell}
%页边距
\usepackage{geometry}
\geometry{left=2cm,right=2cm,top=2cm,bottom=2cm}

%首行缩进两字符 利用\indent \noindent进行控制
\usepackage{indentfirst}
\setlength{\parindent}{2em}

\setromanfont{Songti SC}
%\setromanfont{Heiti SC}

\title{Probability and Mathmatical Statistics \\Homework 14}
\author{冯诗伟161220039}
\date{}
\begin{document}
\maketitle
\section{8.2}
解:
\indent 设桶装油的质量为$X$千克,$X \sim N(\mu,\sigma^2)$,记原假设$H_0$和备择假设$H_1$分别为
\[
    H_0: \mu =10 \quad H_1 :\mu \ne 10
    \]
检验统计量为$\dfrac{\overline X -\mu}{S_n / \sqrt{n-1}}$,当$H_0$成立时,
$T=\dfrac{\overline X -10}{S_n / \sqrt{n-1}}\sim t(n-1)$。则检验的拒绝域为
\[
    W = \left\{ |T|=|\dfrac{\overline X -10}{S_n / \sqrt{n-1}}|\ge t_{\alpha / 2}(n-1) \right\}
    \]
其中,$\overline X=10.06,S_n=\sqrt{S_n^2}=0.7375,n=10,t_{\alpha / 2}(n-1)=t_{0.005}(9)=3.2498,$
检验统计量的观察值$t=0.244\notin W,$所以接受原假设,认为该公司的桶装油质量为10千克。
\section{8.3}
解:
\indent 设这批矿砂的锂含量为$X\%$,$X \sim N(\mu,\sigma^2)$,记原假设$H_0$和备择假设$H_1$分别为
\[
    H_0: \mu =3.25 \quad H_1 :\mu \ne 3.25
    \]
检验统计量为$\dfrac{\overline X -\mu}{S_n / \sqrt{n-1}}$,当$H_0$成立时,
$T=\dfrac{\overline X -3.25}{S_n / \sqrt{n-1}}\sim t(n-1)$。则检验的拒绝域为
\[
    W = \left\{ |T|=|\dfrac{\overline X -3.25}{S_n / \sqrt{n-1}}|\ge t_{\alpha / 2}(n-1) \right\}
    \]
其中,$\overline X=3.252,S_n=\sqrt{S_n^2}=0.01166,n=5,t_{\alpha / 2}(n-1)=t_{0.005}(4)=4.6041,$
检验统计量的观察值$t=0.343\notin W,$所以接受原假设,认为该批矿砂的锂含量为3.25\%。

\section{8.4}
解:
\indent 设这批钢索的断裂强度为$X$千克/平方厘米,$X \sim N(\mu,40^2)$,记原假设$H_0$和备择假设$H_1$分别为
\[
    H_0: \mu =\overline x - 20 \quad H_1 :\mu < \overline x - 20
    \]
检验统计量为$\dfrac{\overline X -\mu}{\sigma / \sqrt{n}}$,当$H_0$成立时,
$U=\dfrac{\overline X -(\overline x - 20)}{\sigma / \sqrt{n}}\sim N(0,1)$。则检验的拒绝域为
\[
    W = \left\{ U=\dfrac{\overline X -(\overline x - 20)}{\sigma / \sqrt{n}} \le -u_{\alpha} \right\}
    \]
其中,$\overline X=\overline x,\sigma = 40,n=9,-u_{\alpha}=-u_{0.01}=-2.33,$
检验统计量的观察值$u=1.5\notin W,$所以接受原假设,认为该批钢索的断裂强度有所提高。

\section{8.6}
解:
\indent 设$A$种小麦的蛋白质含量为$X$,$X \sim N(\mu_1,\sigma^2)$,
设$B$种小麦的蛋白质含量为$Y$,$Y \sim N(\mu_2,\sigma^2)$,
记原假设$H_0$和备择假设$H_1$分别为
\[
    H_0: \mu_1 =\mu_2 \quad H_1 :\mu_1 \ne \mu_2 
    \]
检验统计量为$\sqrt{\dfrac{n_1n_2(n_1+n_2-2)}{n_1+n_2}}\dfrac{(\overline X-\overline Y)-(\mu_1-\mu_2)}{\sqrt{(n_1-1)S_{1}^2+(n_2-1)S_{2}^2}}$,\\
当$H_0$成立时,
$T=\sqrt{\dfrac{n_1n_2(n_1+n_2-2)}{n_1+n_2}}\dfrac{\overline X-\overline Y}{\sqrt{(n_1-1)S_{1}^2+(n_2-1)S_{2}^2}}\sim t(n_1+n_2-2)$。\\
则检验的拒绝域为
\[
    W = \left\{ |T| \ge t_{\alpha /2}(n_1+n_2-2) \right\}
    \]
其中,$n_1 =10,\overline X=14.3,S_1^2=1.62,n_2=5,\overline Y = 11.7,S_2^2 = 0.14,t_{\alpha /2}(n_1+n_2-2)=t_{0.005}(13)=3.0123,$
检验统计量的观察值$t=4.399\in W,$所以接受备择假设,认为良种小麦的蛋白质含量有差异。

\section{8.10}
解:
\indent 设机床甲加工的零件直径为$X$,$X \sim N(\mu,\sigma_1^2)$,
设机床乙加工的零件直径为$Y$,$Y \sim N(\mu,\sigma_2^2)$,
记原假设$H_0$和备择假设$H_1$分别为
\[
    H_0: \sigma_1^2 = \sigma_2^2 \quad H_1 :\sigma_1^2 \ne \sigma_2^2
    \]
检验统计量为$\dfrac{S_1^2\sigma_2^2}{S_2^2\sigma_1^2}$,当$H_0$成立时,
$F=\dfrac{S_1^2}{S_2^2}\sim(n_1-1,n_2-1)$。则检验的拒绝域为
\[
    W = \left\{ F \le F_{1-\alpha/2}(n_1-1,n_2-1) \right\} \cup
        \left\{ F \ge F_{\alpha/2}(n_1-1,n_2-1) \right\}
    \]
其中,$n_1 =8,\overline X =19.925,S_1^2=0.2164,n_2=7,\overline Y = 20,S_2^2 = 0.6298,F_{1-\alpha /2}(n_1-1,n_2-1)=F_{0.975}(7,6)=\dfrac{1}{F_{0.025}(6,7)}=0.1953,F_{\alpha /2}(n_1-1,n_2-1)=F_{0.025}(7,6)=5.70$
检验统计量的观察值$f=0.3436 \notin W,$所以接受原假设,认为甲乙两台机床加工的精度无显著差异。

\end{document}