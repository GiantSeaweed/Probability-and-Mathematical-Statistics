%!TEX program = xelatex
\documentclass[a4papers]{ctexart}
%数学符号
\usepackage{amssymb}
\usepackage{amsmath}
%表格
\usepackage{graphicx,floatrow}
\usepackage{array}
\usepackage{booktabs}
\usepackage{makecell}
%页边距
\usepackage{geometry}
\geometry{left=2cm,right=2cm,top=2cm,bottom=2cm}

%首行缩进两字符 利用\indent \noindent进行控制
\usepackage{indentfirst}
\setlength{\parindent}{2em}

\setromanfont{Songti SC}
%\setromanfont{Heiti SC}

\title{Probability and Mathmatical Statistics}
\author{冯诗伟161220039}
\date{}
\begin{document}
\maketitle
\section{习题2.1}
将3个小球随即放入4个盒子中,设盒子中求的最多个数位X,求X的分布律。\\
\noindent 解:
\indent 设盒子中求的最多个数位X,X是一个随机变量,X的取值范围为1,2,3。\\
\[ P(X=1) = \frac{A_4^3}{4^3}=\frac{24}{64}=\frac{3}{8} \]
\[ P(X=2) =\frac{C_3^2 A_4^2}{4^3} = \frac{36}{64}=\frac{9}{16} \]
\[P(X=3) = \frac{C_4^1}{4^3} = \frac{4}{64}=\frac{1}{16} \]
所以X的分布律为:\\
\begin{center}
\begin{tabular}{c|ccc}
X & 1 & 2 & 3 \\
\hline
P & 3/8 & 9/16 & 1/16
\end{tabular}
\end{center}

\section{习题2.3}
问C取何值时一下数列称为概率分布律:\\
(1) $p_k$ = C$\left( \dfrac{2}{3} \right)^2$, $k$ = 1,2,3;\\
(2) $p_k$ = C$\dfrac{\lambda^k}{k!}$, $k$ = 1,2,3,…… \\
\noindent 解:\\
(1) 由公式$\sum_{k=1}^3 C\left(\dfrac{2}{3}\right)^k = 1$,
得$C\times\dfrac{38}{27} = 1$,所以$C = \dfrac{27}{38}。$\\
(2) 由公式$\sum_{k=1}^\infty C\dfrac{\lambda^k}{k!} = 1$,
得$C\sum_{k=1}^\infty \dfrac{\lambda^k}{k!} = C\*e^\lambda = 1$,
所以$C = \dfrac{1}{e^\lambda}$。\\

\section{习题2.19}
设离散型随机变量X的分布律为:\\
\begin{center}
\begin{tabular}{c|ccccc}
X & -2 & -1/2 & 0 & 1/2 & 4 \\
\hline
P & 1/8 & 1/4 & 1/8 & 1/6 & 1/3
\end{tabular}
\end{center}
求下列随机变量的分布律:(1) Y = 2X; (2) Y = $X^2$; (3) $Y = {\rm \sin}\left(\dfrac{\pi}{2}X\right)$。\\
解:(1)
\begin{center}
\begin{tabular}{c|ccccc}
Y & -4 & -1 & 0 & 1 & 8 \\
\hline
P & 1/8 & 1/4 & 1/8 & 1/6 & 1/3
\end{tabular}
\end{center}
(2)
\begin{center}
\begin{tabular}{c|cccc}
Y & 0 & 1/4 & 4 & 16 \\
\hline
P & 1/8 & 5/12 & 1/8 & 1/3 
\end{tabular}
\end{center}
(3)
X和Y的取值关系如下表,Y只有三种取值$-\dfrac{\sqrt{2}}{2}$,$0$,$\dfrac{\sqrt{2}}{2}$。\\
\begin{center}
\begin{tabular}{c|ccccc}
X & -2 & -1/2 & 0 & 1/2 & 4 \\
\hline
Y & 0 & $-\dfrac{\sqrt{2}}{2}$ & 0 & $\dfrac{\sqrt{2}}{2}$ & 0
\end{tabular}
\end{center}
所以Y的分布律为:\\
\begin{center}
\begin{tabular}{c|ccccc}
Y & $-\dfrac{\sqrt{2}}{2}$ & 0 & $\dfrac{\sqrt{2}}{2}$ \\
\hline
P & 1/4 & 7/12 & 1/6 
\end{tabular}
\end{center}

\section{习题3.3}
设随机变量X和Y相互独立,且X和Y的概率分布为\\

\begin{minipage}{0.5\textwidth}  %两个表格并排
\begin{tabular}{c|cccc}
\hline
X & 0 & 1 & 2 & 3 \\
\hline
P & 1/2 & 1/4 & 1/8 & 1/8 \\
\hline
\end{tabular}
\end{minipage}
\hfil
\begin{minipage}{0.5\textwidth}
\begin{tabular}{c|ccc}
\hline
Y & -1 & 0 & 1 \\
\hline
P & 1/4 & 1/2 & 1/4 \\
\hline
\end{tabular} 
\end{minipage}
\\[2mm]                 %可加入任意间距的空白行
\indent 请计算概率$P(X+Y=2)$。

\noindent 解:
\begin{alignat*}{2}        % alignat* 使得公式没有序号
P(X+Y=2) &= P(X=1,Y=1)+P(X=2,Y=0)+P(X=3,Y=-1)\\
&= P(X=1)P(Y=1)+P(X=2)P(Y=0)+P(X=3)P(Y=-1)\\
&= \dfrac{1}{4}\times\dfrac{1}{4}+ \dfrac{1}{8}\times\dfrac{1}{2}+ \dfrac{1}{8}\times\dfrac{1}{4}\\
&= \dfrac{1}{16}+\dfrac{1}{16}+\dfrac{1}{32}\\
&= \dfrac{5}{32}
\end{alignat*}

\section{蓄水池抽样}
有一系列的数据流经某系统。我们希望对该数据流进行采样,希望能都从该数据流中
采样一个数据,使得该数据等可能地为所有已经流经该系统的数据中的一个。假设我们不知道
数据流中数据的个数,同时也不保存已经流经系统的数据。\\
\indent 考虑以下算法:当第一个数据经过时,我们将其存放在内存中。
当第k个数据经过时,我们以$\dfrac{1}{k}$的概率用其替代内存中的数据。\\
(1)证明该算法的有效性。\\ 
(2)如果替代的概率为$\dfrac{1}{2}$,求内存中存放的是哪一个数据的分布。\\
\noindent 解:\\
(1)设事件$A_i$表示第i个数据替代了内存中的数据,事件$B_i$表示第i个数据最终占据内存,
设共流经n个数据,则
\begin{alignat*}{2}
    P(B_i)
    &=P(A_i \overline{A_{i+1}}\, \overline{A_{i+2}}\cdots\overline{A_n})\\
    &=P(A_i)P(\overline{A_{i+1}})P(\overline{A_{i+1}}\cdots P(\overline{A_n})\\
    &=\dfrac{1}{i} \dfrac{i}{i+1} \dfrac{i+1}{i+2}…\dfrac{n-1}{n}\\
    &=\dfrac{1}{n}
\end{alignat*}
所以第i个数据最终占据内存的概率和它到来的次序没用关系,得证。\\
(2)设事件$A_i$表示第i个数据替代了内存中的数据,事件$B_i$表示第i个数据最终占据内存,
设共流经n个数据,则$P(A_i)=\dfrac{1}{2}$。\\
\begin{alignat*}{2}
P(B_i)
&=P(A_i \overline{A_{i+1}}\, \overline{A_{i+2}}\cdots\overline{A_n})\\
&=\begin{matrix}\underbrace{\dfrac{1}{2} \dfrac{1}{2} \dfrac{1}{2} \cdots \dfrac{1}{2}}\\ n-i+1个 \end{matrix}\\
&=\dfrac{1}{2^{n-i+1}}
\end{alignat*}
且满足\[
    P(B_i)\ge 0,\,\sum_{i=1}^\infty P(B_i)
    =\lim_{n \to \infty}\sum_{i=1}^n \dfrac{1}{2^{n-i+1}} 
    =\lim_{n \to \infty}\sum_{i=1}^n \dfrac{1}{2^i} 
    = 1
\]
\section{补充3 }
考虑取值为正整数的随机变量X,其分布律为$P(X=i)=\dfrac{6}{\pi}i^{-2}$
(注意$\sum_{i=1}^\infty i^{-2}=\dfrac{\pi^2}{6}$)。分析X的期望值。\\
\noindent 解:\\
\[E[X]=\sum_{i=1}^\infty iP(X=i) = \sum_{i=1}^\infty \dfrac{6}{\pi}i^{-1}\]
\indent 其中级数$\sum_{i=1}^\infty\dfrac{6}{\pi}i^{-1}$发散,所以X的期望不存在。

\section{n人牵手}
考虑$n$个人玩一个$n$轮的游戏。在第$i$($i=1,2,…,n$)轮游戏中,从剩下的
$2n-2(i-1)$只手中随机挑选2只手进行牵手。求形成环的个数的期望值。(某人左右手牵在一起也算形成一个环)。\\
\noindent 解:\\
\indent 每轮游戏去掉两只手,即去掉“一个人”:若这两只手是同一个人的,则成了一个环,且这个人再也不参与之后的游戏,相当于去掉了一个人;
若这两只手不是同一个人的,则没有成环,但这两个人相当于合并成了一个人,也算是去掉了一个人。\\
\indent 构造指示变量$A_i$,
\[ A_i = \begin{cases}
    0, & \mbox{if 第}i\mbox{轮没有成环}\\
    1, & \mbox{if 第}i\mbox{轮有成环}\\
\end{cases} \]
\indent 设最终形成了X个环,
\begin{alignat*}{2}
     E[X] &= \sum_{i=1}^n P(A_i=1)\\
     &= \sum_{i=1}^n \dfrac{n-i+1}{C_{2n-2(i-1)}^2}\\
     &= \sum_{i=1}^n \dfrac{1}{2n-2i+1}
\end{alignat*}

\section{Jensen不等式}
设$f$为下凸函数,即对于任意$x_1,x_2$且$0\le\lambda\le 1$,
\[f(\lambda x_1 + (1-\lambda)x_2)\le\lambda f(x_1)+(1-\lambda)f(x_2)\]
\indent 假设X为只有有限个取值的离散型随机变量,求证
\[E[f(X)]\ge f(E[X])\]
\indent (注:该不等式对连续型随机变量同样适用)\\
\noindent 解:\\
首先证明对于下凸函数$f$,任意的$x_1,x_2\cdots x_n,p_1,p_2\cdots,p_n,0\le p_i \le 1,\sum_{i=1}^n p_i=1$,有\\
\[ f(\sum_{i=1}^n p_i x_i)\le \sum_{i=1}^n p_i f(x_i)  \]
使用数学归纳法,对$n$进行归纳。\\
(1)当$n=2$时,由题目条件已知成立。\\
(2)假设当$n \le k$时该不等式均成立,当$n=k+1$时,\\
\begin{alignat*}{2}
f(\sum_{i=1}^{k+1} p_i x_i) 
&= f(p_{i+1} x_{i+1} + \sum_{i=1}^k p_i x_i)\\
&= f\left( p_{k+1} x_{k+1} + (1-p_{k+1})\dfrac{\sum_{i=1}^k p_i x_i}{1-p_{k+1}} \right)\\
&\le p_{k+1} f(x_{k+1}) + (1-p_{k+1}) f\left(\dfrac{\sum_{i=1}^k p_i x_i}{1-p_{k+1}}\right) \cdots \cdots 由题目条件\\
&\le p_{k+1} f(x_{k+1}) + (1-p_{k+1}) \dfrac{\sum_{i=1}^k p_i f(x_i)}{1-p_{k+1}}\quad\cdots \cdots 由归纳假设\\
&= \sum_{i=1}^{k+1} p_i f(x_i) 
\end{alignat*}
综合(1)(2),对于下凸函数$f$,任意的$x_1,x_2\cdots x_n,p_1,p_2\cdots,p_n,0\le p_i \le 1,\sum_{i=1}^n p_i=1$,有\\
\[ f(\sum_{i=1}^n p_i x_i)\le \sum_{i=1}^n p_i f(x_i)  \]\\
设离散型随机变量X的取值范围是$x_1,x_2\cdots x_n$,分布律是$P(x_i)=p_i,其中0\le p_i \le 1,\sum_{i=1}^n p_i=1$。则有\\
\begin{alignat*}{2}
    f(E[X])&= f(\sum_{i=1}^n p_i x_i)\\
           &\le \sum_{i=1}^n p_i f(x_i) \\
           &= E[f(x)]\\
\end{alignat*}
得证。
\end{document}