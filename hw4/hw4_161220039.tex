%!TEX program = xelatex
\documentclass[a4papers]{ctexart}
%数学符号
\usepackage{amssymb}
\usepackage{amsmath}
%表格
\usepackage{graphicx,floatrow}
\usepackage{array}
\usepackage{booktabs}
\usepackage{makecell}
%页边距
\usepackage{geometry}
\geometry{left=2cm,right=2cm,top=2cm,bottom=2cm}

%首行缩进两字符 利用\indent \noindent进行控制
\usepackage{indentfirst}
\setlength{\parindent}{2em}

\setromanfont{Songti SC}
%\setromanfont{Heiti SC}

\title{Probability and Mathmatical Statistics \\Homework 4}
\author{冯诗伟161220039}
\date{}
\begin{document}
\maketitle
\section{1.36}
某单位有100台电话分机,每台分机有5\%的时间使用外线通话,若每台分机使用外线是独立的,
问该单位要设立多少条外线,才能以90\%以上的概率保证各分机使用外线不被占线。\\
解:\\
设同时有$X$台电话分机要使用外线。$X$可近似看作服从$\lambda=100\times 5\%=5$的泊松分布。\\
设该单位至少设立n条外线,由
\[
    \arg \min_{n \in \mathbb{N}}\left[P(X\le n)\ge 90\% \right]=8
\]
有该单位要设立8条外线,才能以90\%以上的概率保证各分机使用外线不被占线。
\section{2.7}
设有甲乙两种颜色和味觉都极其相似的名酒各4杯,若从中挑4杯能将甲酒全部挑出来,
算是试验成功一次。\\
(1)某人随机地去猜,问他实验成功一次的概率是多少?\\
(2)某人声称他通过品尝可区分这两种酒,他独立试验10次成功3次。利用小概率事件原理推断,
他是猜对的,还是确有区分能力。\\
解:\\
(1)设事件X为实验成功一次。
\[P(X)=\dfrac{1}{\dbinom{8}{4}} = \dfrac{1}{70}\]
(2)设事件Y为独立试验10次成功3次。
\[P(Y)=\dbinom{10}{3}(P(X))^3(1-P(X))^7=0.0316\%\]
事件Y的发生概率非常小,他应该确有分辨能力。\\

\section{2.8}
已知每天到达某港口的油船数X服从参数为2.5的泊松分布,而港口的服务能力最多只有3只船,
如果一天中到达港口的游船多于3只,则超过3只的油船必须转港。\\
(1)求一天中必须有油船转港的概率。\\
(2)求一天中最大可能到达港口的油船数及其概率。\\
(3)问服务能力提高到多少只油船的时候,能使到达油船以90\%的概率得到服务。\\
解:\\
(1)\[P(X\ge4)=1-P(X\le3)=1-0.0821-0.2052-0.2565-0.2138=0.2424\]
(2)设一天中最大可能到达港口的油船数为$i$,则有\\
\[\begin{cases}
    P(X=i)\ge P(X=i+1)\\
    P(X=i)\ge P(X=i-1)\\
\end{cases}\]
解得\[1.5\le i \le 2.5\]
则一天中最大可能到达港口的油船数为$2$,其概率为$P(X=2)=0.2052$。\\
(3)即求$\arg\min_{n\in \mathbb{N}}\left[P(X \le n)\le 90\%\right]$。\\
求得$n=5$。

\section{负二项分布}
解:独立抛硬币$r$次,第$r$次确定为正面向上,前$r-1$次中有$k$次正面向上,
$r-k$次为反面向上。\\
\begin{alignat*}{2}
P(X=r) &=p \dbinom{r-1}{k-1} p^{k-1} (1-p)^{r-k} \\
&= p^k \dbinom{r-1}{k-1}  (1-p)^{r-k},r=k,k+1,\cdots\\
\end{alignat*}
其中,
\[P(X=r)\ge 0, \, r=k,k+1,\cdots\]
\begin{alignat*}{2}
\sum_{r=k}^\infty P(X=r) 
&= \sum_{r=k}^\infty p^k \dbinom{r-1}{k-1}  (1-p)^{r-k} \cdots\cdots 设t=r-k\\
&= p^k \sum_{t=0}^\infty  \dbinom{k+t-1}{t}  (1-p)^{t} \\
&= p^k \sum_{t=0}^\infty  \dfrac{(k+t-1)(k+t-2)\cdots(k)}{t!} (1-p)^t\\
&= p^k \sum_{t=0}^\infty (-1)^t \dfrac{(-k)(-k-1)\cdots(-k-(t-1))}{t!} (1-p)^t\\ 
&= p^k \sum_{t=0}^\infty \dbinom{t}{-k} (p-1)^t · 1^{-k-t}\\
&= p^k p^{-k}\\
&= 1
\end{alignat*}    
\section{补充3}
$假设X和Y相互独立,且分别服从参数为p和q的几何分布,求下列的值:$\\
(1)$P(X=Y)$\\
(2)$P(min(X, Y)=k)。想一想min(X,Y)的物理含义。$\\
解:\\
(1)
\begin{alignat*}{2}
P(X=Y) &= \sum_{k=1}^\infty P(X=k,Y=k)\\
&= \sum_{k=1}^\infty P(X=k)P(Y=k)\\
&= \sum_{k=1}^\infty (1-p)^{k-1}p(1-q)^{k-1}q\\
&= pq\sum_{k=1}^\infty [(1-p)(1-q)]^{k-1}\\
&= \dfrac{pq}{p+q-pq}\\
\end{alignat*}
(2)
$min(X,Y)$是指直到$X$和$Y$中任意一个成立所进行的实验次数。
令$W$表示直到$X$和$Y$中任意一个成立所进行的实验次数,
则有$W~G(1-(1-p)(1-q)),即W~G(p+q-pq)$。则有
\begin{alignat*}{2}
P(min(X, Y)=k) &= P(W)\\
&= \sum_{k=1}^\infty [1-(p+q-pq)]^{k-1}(p+q-pq)\\
&= (1-p)^{k-1}(1-q)^{k-1}(p+q-pq)\\
\end{alignat*}
\section{非匀质硬币}
抛一枚非匀质硬币,其正面向上的概率计为$p$,但是数值未知。请给出一种方法,
能够利用这枚硬币生成无偏的随机比特(即是0或是1的概率各是0.5),并保证所需抛硬币的
次数的期望不超过$\dfrac{1}{[p(1-p)]}$。\\
解:\\
方法:\\
\indent 连续抛两次硬币,若是“正反”则生成1,若是“反正”则生成0,若是“正正”或“反反”则重抛一次。\\
说明:\\
\indent 连续抛两次硬币,若是“正反”则记为事件$X_1$,若是“反正”则记为事件$X_2$,
若是“正正”则记为事件$X_3$,若是“反反”则记为事件$X_4$。\\
\indent 因为$P(X_1)=P(X_2)=p(1-p)$,所以生成的是无偏的随机比特。\\
\indent 记事件Y为成功生成无偏的随机比特。
\[P(Y)=P(X_1)+P(X_2)=2p(1-p)\]
\[
P(\overline Y)=P(X_3)+P(X_4)=p^2+(1-p)^2=2p^2-2p+1
\]
\indent 所以有
\[E(Y)=\dfrac{1}{2p(1-p)}\]
\indent 因为事件Y的每一次试验都需要抛两次硬币,所以题目中为生成无偏的随机比特保证所需抛硬币的
次数的期望为$2E(Y)=\dfrac{1}{p(1-p)}$。
\end{document}