%!TEX program = xelatex
\documentclass[a4papers]{ctexart}
%数学符号
\usepackage{amssymb}
\usepackage{amsmath}
\usepackage{amsthm}
%表格
\usepackage{graphicx,floatrow}
\usepackage{array}
\usepackage{booktabs}
\usepackage{makecell}
%页边距
\usepackage{geometry}
\geometry{left=2cm,right=2cm,top=2cm,bottom=2cm}

%首行缩进两字符 利用\indent \noindent进行控制
\usepackage{indentfirst}
\setlength{\parindent}{2em}

\setromanfont{Songti SC}
%\setromanfont{Heiti SC}

\title{Probability and Mathmatical Statistics \\Homework 5}
\author{冯诗伟161220039}
\date{}
\begin{document}
\maketitle
\section{4.4}
已知随机变量$X$和$Y$独立同分布,且X的概率分布为$P(X=1)=1-P(X=2)=\dfrac{2}{3}$。
记$U=max(X,Y),V=min(X,Y)$,求:\\
\indent (1)$(U,V)$的概率分布;(2)$E(U)$和$E(V)$;(3)$cov(U,V)$。\\
解:\\
(1) $P(X=1)=P(Y=1)=\dfrac{2}{3},P(X=2)=P(Y=2)=\dfrac{1}{3}$。\\
$(U,V)的取值范围为(1,1),(2,1),(2,2)。$\\
$P((U,V)=(1,1))=P(X=1,Y=1)=P(X=1)P(Y=1)=\dfrac{4}{9},$\\
$P((U,V)=(2,1))=P(X=2,Y=1)+P(X=1,Y=2)=P(X=2)P(Y=1)+P(X=1)P(Y=2)=\dfrac{4}{9},$\\
$P((U,V)=(2,2))=P(X=2,Y=2)=P(X=2)P(Y=2)=\dfrac{1}{9}$,\\
所以$(U,V)$的分布律如下:\\
\begin{center}
\begin{tabular}{c|ccc}
\hline
$(U,V)$ & $(1,1)$ & $(2,1)$ & $(2,2)$ \\
\hline
$P$ & $\frac{4}{9}$& $\frac{4}{9}$ & $\frac{1}{9}$\\
\hline
\end{tabular}
\end{center}
(2)\,$U$的$V$分布律如下:\\[2mm]
\begin{minipage}{0.4\textwidth}  %两个表格并排
\begin{tabular}{c|cc}
\hline
$U$ &  1 & 2  \\
\hline
$P$ & $\frac{4}{9}$ & $\frac{5}{9}$ \\
\hline
\end{tabular}
\end{minipage}
\hfil
\begin{minipage}{0.8\textwidth}
\begin{tabular}{c|cc}
\hline
$V$ & 1 & 2  \\
\hline
$P$ & $\frac{8}{9}$ & $\frac{1}{9}$ \\
\hline
\end{tabular} 
\end{minipage}
\\[2mm]
$E(X)=1\times\dfrac{4}{9}+2\times\dfrac{5}{9}=\dfrac{14}{9}$,
$E(Y)=1\times\dfrac{8}{9}+2\times\dfrac{1}{9}=\dfrac{10}{9}$。\\[2mm]
(3)
\begin{center}
\begin{tabular}{c|ccc}
\hline
$UV$ &  1 & 2 & 4  \\
\hline
$P$ & $\frac{4}{9}$ & $\frac{4}{9}$& $\frac{1}{9}$ \\
\hline
\end{tabular}
\end{center}

\begin{alignat*}{2}
 cov(U,V)&=E(UV)-E(U)E(V)\\
 &=\sum_{i=1,2,4}iP(UV=i)-E(U)E(V)\\
 &=\dfrac{16}{9}-\dfrac{14}{9}\times\dfrac{10}{9}\\
 &=\dfrac{4}{81}
\end{alignat*}
\section{4.19}
设$X$和$Y$独立,且都服从泊松分布,已知$E(X)=1,E(Y)=2$,请计算$E((X+Y)^2)$。\\
解:\\
由题意,$X~P(1),Y~P(2)$。又因为$E(X^2)=1^2+1,E(Y^2)=2^2+2$,有
\begin{alignat*}{2}
    E((X+Y)^2)&=E(X^2+Y^2+2XY)\\
    &=E(X^2)+E(Y^2)+2E(XY)\\
    &=E(X^2)+E(Y^2)+2E(X)E(Y)\\
    &=(1^2+1)+(2^2+2)+2\times1\times2\\
    &=12\\
\end{alignat*}
\section{4.23}
    设$f(x),0\le x < +\infty$,是一个单调非减函数,且$f(x)>0$。对于随机变量$X$,
若$E[f(x)]<\infty$,则证明对于任意的$x>0,P(|X|\ge x)\le \dfrac{1}{f(x)} E[f(|X|)]$。
\begin{proof}
由于$f(x)$是一个单调非减函数,所以\[P(|X|\ge x)=P(f(|X|)\ge f(x))\]
由马尔可夫不等式,有\[P(f(|X|)\ge f(x))\le \dfrac{1}{f(x)} E[f(|X|)]\]
综上对于任意的$x>0,P(|X|\ge x)\le \dfrac{1}{f(x)} E[f(|X|)]$。
\end{proof} %proof环境可以有证毕符号!!

\section{补充2}
假设$X$和$Y$独立,$X~G(p),Y~G(q)$,求下列值。\\
(1)$E[max(X,Y)].$(至少用两种方法)\\
(2)$E[X|X\le Y].$\\
解:\\
(1)
($a$)方法一:按照定义
\begin{align*}
    P(max(X,Y)=k)&=P(X=k,Y\le k)+P(Y=k,X< k)\quad(防止X=Y被重复计算)\\
    &=P(X=k)P(Y\le k)+P(Y=k)P(X < k)\\
    &=P(X=k)\left(1-P(Y > k)\right)+P(Y=k)\left(1-P(X\ge k)\right)\\
    &=(1-p)^{k-1}p(1-(1-q)^k)+(1-q)^{k-1}q(1-(1-p)^{k-1})\\
    &=(1-p)^{k-1}p+(1-q)^{k-1}q-(p+q-pq)(1-p)^{k-1}(1-q)^{k-1}
\end{align*}

\begin{align*}
    E[max(X,Y)]&=\sum_{k=1}^\infty kP(max(X,Y)=k)\\
    &=\sum_{k=1}^\infty k(1-p)^{k-1}p +\sum_{k=1}^\infty k(1-q)^{k-1}q-\sum_{k=1}^\infty k(p+q-pq)(1-p)^{k-1}(1-q)^{k-1}\\
    &=\dfrac{1}{p}+\dfrac{1}{q}-\dfrac{1}{p+q-pq}
\end{align*}
($b$)方法二:$X,Y$中总有一个等于$max{X,Y}$,另一个等于$min{X,Y}$。
作业4中曾求得$min(X,Y)~G(p+q-pq)$,所以$E(min(X,Y))=\dfrac{1}{p+q-pq}$。
\begin{align*}
    E(max(X,Y))&=E(X)+E(Y)-E(min(X,Y))\\
    &=\dfrac{1}{p}+\dfrac{1}{q}-\dfrac{1}{p+q-pq}
\end{align*}
(2)
\begin{align*}
    E[X|X\le Y]&=\sum_{k=1}^\infty kP(X=k|X\le Y)\\
    &=\sum_{k=1}^\infty k\dfrac{P(X=k,X\le Y)}{P(X\le Y)}\\
\end{align*}
\indent 其中
\begin{align*}
    P(X=k,X\le Y)=(1-p)^{k-1}p(1-q)^{k-1}
\end{align*}
\begin{align*}
    P(X\le Y)&=\sum_{i=1}^\infty P(X=i,Y\ge i)\\
    &=\sum_{i=1}^\infty (1-p)^{i-1}p(1-q)^{i-1}\\
    &=\dfrac{p}{p+q-pq}
\end{align*}
\indent 所以
\begin{align*}
    E[X|X\le Y]&=\sum_{k=1}^\infty k\dfrac{P(X=k,X\le Y)}{P(X\le Y)}\\
    &=(p+q-pq)\sum_{k=1}^\infty (1-p)^{k-1}(1-q)^{k-1}\\
    &=\dfrac{1}{p+q-pq}
\end{align*}
\section{补充3}
设$\pi$为$[n]={1,2,3,\cdots,n}$的一个置换。若$\pi(i)=i$,则称$i$为$\pi$的一个不动点。
从$n!$个置换中任取一个置换,求不动点个数的方差。\\
解:\\
定义随机变量$X$表示不动点的个数,同时定义指示变量$X_i(i=1,2,3,\cdots,n)$,
\[ X_i=\begin{cases}
    1,& \pi(i)=i\\
    0,& \pi(i)\ne i\\
\end{cases}\]
$P(X_i=1)=\dfrac{1}{n}$,则\\
\begin{alignat*}{2}
    E(X)&=E(\sum_{i=1}^n X_i)\\
    &=\sum_{i=1}^nE(X_i)\\
    &=1
\end{alignat*}

\begin{align*}
    E(X^2)&=E((\sum_{i=1}^n X_i)^2)\\
    &=E\left(\sum_{i=1}^n X_i^2 + 2\sum_{1\le i<j\le n} X_i X_j\right)\\
    &=\sum_{i=1}^n E(X_i^2)+2\sum_{1\le i<j\le n}E(X_i X_j)\\
    &=\sum_{i=1}^n P(X_i=1)+2\sum_{1\le i<j\le n}P(X_i=1,X_j=1)\\
    &=n\times\dfrac{1}{n}+2\times\dfrac{n(n-1)}{2}\times\dfrac{(n-2)!}{n!}\\
    &=2
\end{align*}
得
\[D(X)=E(X^2)-[E(X)]^2=2-1=1\]
\section{补充4}
不停地抛一枚均匀的骰子,直至出现一双连续的6。求所抛次数的期望值。\\
解:\\
\indent 定义X为出现一双连续的6所抛次数,同时定义两个随机变量Y、Z。\\
\[ Y=\begin{cases}
    1,& 第一次是6\\
    0,& 第一次不是6\\
\end{cases}\]
\[ Z=\begin{cases}
    1,& 第二次是6\\
    0,& 第二次不是6
\end{cases}\]
\begin{align*}
    E(X)&=E(X|Y=1)P(Y=1)+E(X|Y=0)P(Y=0)\\
    &= \left[ E(X|Y=1,Z=1)P(Z=1)+ E(X|Y=1,Z=0)P(Z=0) \right] P(Y=1)+(1+E[X])P(Y=0)\\
    &=[2\times\dfrac{1}{6}+(2+E(X))\times\dfrac{5}{6}]\times\dfrac{1}{6}+(1+E(X))\times\dfrac{5}{6}\\
\end{align*}
\indent 解得\quad$E(X)=42$。


\section{补充5}
某只股票每天的股票以$p$的概率变成原来的$r>1$倍,以$q=1-p$的概率变成原来的$1/r$倍。
假定股票的初始价格为$1$元每股,计算$d$天之后股票价格的期望和方差。\\
解:\\
\indent 设第$i$天的股票价格为$X_i$,
\begin{align*}
    E(X_i)&=pE(X_i|第i-1天涨价)+(1-p)E(X_i|第i-1天降价)\\
    &=prE(X_{i-1})+\dfrac{1-p}{r}E(X_{i-1})\\
    &=\left(pr+\dfrac{1-p}{r}\right)E(X_{i-1})
\end{align*}
\indent 因为$E(X_1)=1$,所以$E(X_d)=(pr+\dfrac{1-p}{r})^{d-1}$。
\begin{align*}
    E(X_i^2)&=pE(X_i^2|第i-1天涨价)+(1-p)E(X_i^2|第i-1天降价)\\
    &=pE(X_{i-1}^2\times r^2)+E\left(X_{i-1}^2\times(\dfrac{1-p}{r})^2\right)\\
    &=\left[p r^2+\dfrac{1-p}{r^2}\right]E(X_{i-1})
\end{align*}
\indent 因为$E(X_1^2)=1$,所以$E(X_d^2)=\left[p r^2+\dfrac{1-p}{r^2}\right]^{d-1}$。
所以有
\[
    D(X_d)=E(X^2)-[E(X)]^2
    =\left[p r^2+\dfrac{1-p}{r^2}\right]^{d-1}-\left(pr+\dfrac{1-p}{r}\right)^{2d-2}
    \]

\section{补充6}
设我们抛了$n$次一枚均匀硬币从而获得了$n$个随机比特。考虑这些比特可构成$n(n-1)/2$个比特对。
对每一个比特对进行异或操作,记结果为$Y_i,i=1,2,\cdots,n(n-1)/2$。定义$Y$为$Y_i=1$的个数。\\
(1)求$Y_i$的分布律。\\
(2)证明$Y_i,i=1,2,\cdots,n(n-1)/2$并不是相互独立的。\\
(3)证明$E(Y_iY_j)=E(X_i)E(X_j)$。\\
(4)计算$D(Y)$。\\
(5)为$P(|Y-E(Y)|\ge n)$建立一个上界。\\
解:\\
(1)
\[
    P(Y_i=1)=P(比特对为(1,1))+P(比特对为(0,0))=\dfrac{1}{2}\times\dfrac{1}{2}+\dfrac{1}{2}\times\dfrac{1}{2}=\dfrac{1}{2}\]
\[  P(Y_i=0)=P(比特对为(1,0))+P(比特对为(0,1))=\dfrac{1}{2}\times\dfrac{1}{2}+\dfrac{1}{2}\times\dfrac{1}{2}=\dfrac{1}{2}\\
\]
$Y_i$的分布律如下:\\
\begin{center}
\begin{tabular}{c|cc}
\hline
$Y_i$ & 0 & 1 \\
\hline
$P$ & $\frac{1}{2}$ &$\frac{1}{2}$\\
\hline
\end{tabular}
\end{center}
(2)
\begin{proof}
    取特殊值$Y_i=1,i=1,2,\cdots,\binom{n}{2}$,
\[
    \prod_{i=1}^{\binom{n}{2}} P(Y_i=1)=1 
    \]
但是$P\left((\prod_{i=1}^{\binom{n}{2}} Y_i)=1\right)$必定等于0,
因为任取三个比特,必有两个相等,则存在$i,1\le i \le {\binom{n}{2}}$,使得$Y_i=0$。
$所以Y_i,i=1,2,\cdots,n(n-1)/2$并不是相互独立的。
\end{proof}
\noindent(3)
\begin{proof}
\begin{align*}
    E(Y_iY_j)&=0\times \sum_{1\le i < j\le n(n-1)/2} P(Y_iY_j=0)+1\times \sum_{1\le i < j\le n(n-1)/2}P(Y_iY_j=1)\\
    &=\sum_{1\le i < j\le n(n-1)/2}P(Y_iY_j=1)\\
    &=\sum_{1\le i < j\le n(n-1)/2}P(Y_i=1,Y_j=1)
\end{align*}
\indent 当$Y_i,Y_j$的两个比特对是由4个比特生成时,$Y_i,Y_j$是独立的,
$P(Y_i=1,Y_j=1)=P(X_i=1)P(Y_i=1)。$\\
\indent 当$Y_i,Y_j$的两个比特对是由3个比特生成时,记这三个比特为$a,b,c$。不失一般性,
设$Y_i=a\oplus b,Y_j=a\oplus c$。在$(a,b,c)$的8种情况中,
有2种情况使得$Y_i=1$且$Y_j=1$,$P(Y_i=1,Y_j=1)=\dfrac{1}{4}$;
有4种情况使得$Y_i=1$,$P(Y_i=1)=\dfrac{1}{2}$;
有4种情况使得$Y_j=1$,$P(Y_j=1)=\dfrac{1}{2}$;
所以$P(Y_i=1,Y_j=1)=P(X_i=1)P(Y_i=1)。$\\
\indent 所以
\begin{align*}
    E(Y_iY_j)&=\sum_{1\le i < j\le n(n-1)/2}P(Y_i=1,Y_j=1)\\
    &=\sum_{1\le i < j\le n(n-1)/2}P(Y_i=1)P(Y_j=1)\\
    &=\sum_{i=1}^n P(Y_i=1) \sum_{j=1}^n P(Y_j=1) \quad (i\ne j)\\
    &=E(Y_i)E(Y_j)
\end{align*}
\end{proof}

\noindent(4)
\begin{align*}
    D(Y)&=D\left(\sum_{i=1}^{n(n-1)/2} Y_i\right)\\
    &=\sum_{i=1}^{n(n-1)/2}D(Y_i)+2\sum_{1\le i<j\le \frac{n(n-1)}{2}}cov(Y_i,Y_j)\\
    &=\sum_{i=1}^{n(n-1)/2}D(Y_i)+2\sum_{1\le i<j\le \frac{n(n-1)}{2}}[E(Y_i,Y_j)-E(Y_i)E(Y_j)]\\
    &=\sum_{i=1}^{n(n-1)/2}[E(Y_i^2)-(E(Y_i))^2]\\
    &=\sum_{i=1}^{n(n-1)/2}[(1^2\times\dfrac{1}{2})-(\dfrac{1}{2})^2]\\
    &=\dfrac{n(n-1)}{2}\times\dfrac{1}{4}\\
    &=\dfrac{n(n-1)}{8}
\end{align*}

\noindent(5)
使用切比雪夫不等式,
\[
    P\left(|Y-E(Y)|\ge n\right)\le\dfrac{D(Y)}{n^2}=\dfrac{\dfrac{n(n-1)}{8}}{n^2}=\dfrac{n-1}{8n}
    \]
\end{document}